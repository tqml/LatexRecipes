\documentclass[final,twoside,openright]{book} 
% Useful Options:
% - draft vs. final
% - twocolumn: Use a two column layout
% - twoside: Formats for printing as a book


 % Use an extension of the original Computer Modern font to minimize the use of bitmapped letters.
\usepackage{lmodern}       
% Determines font encoding of the output. Font packages have to be included before this line.
\usepackage[T1]{fontenc}    
% Determines encoding of the input. All input files have to use UTF8 encoding.
\usepackage[utf8]{inputenc} 

%--------------------------------------
\usepackage[ngerman]{babel} %German-specific commands
%--------------------------------------


% Extended LaTeX functionality is enables by including packages with \usepackage{...}.

% Extended typesetting of mathematical expression.
\usepackage{amsmath}     
\usepackage{amssymb}    
\usepackage{mathtools}  
\usepackage{mhchem}

\usepackage{microtype}  % Small-scale typographic enhancements.
\usepackage[inline]{enumitem} % User control over the layout of lists (itemize, enumerate, description).
\usepackage{siunitx}
\usepackage{multirow}   % Allows table elements to span several rows.
\usepackage{booktabs}   % Improves the typesettings of tables.
\usepackage{subcaption} % Allows the use of subfigures and enables their referencing.
\usepackage{nag}       % Issues warnings when best practices in writing LaTeX documents are violated.
\usepackage{todonotes} % Provides tooltip-like todo notes.
\usepackage{epigraph}
\usepackage[acronym,toc]{glossaries} 
\usepackage{graphicx}
\usepackage{csquotes}
\usepackage{lipsum}

%************************************
% BiblioGraphy Settings
%************************************
\usepackage[style=verbose-ibid,backend=bibtex]{biblatex}
\bibliography{references}





%************************************
% PDF Meta Data Settings
%************************************
% Enables cross linking in the electronic document version. This package has to be included second to last.
\usepackage[hidelinks]{hyperref}  

% Set PDF document properties
\makeatletter
\AtBeginDocument{
  \hypersetup{
    pdfpagelayout   = TwoPageRight,
    pdftitle = {\@title},
    pdfauthor = {\@author},
    pdfsubject      = {Subject},              
    pdfkeywords     = {a, list, of, keywords} 
  }
}
\makeatother


%************************************************************
% ToDo: Set thesis title and author
%************************************************************

\newcommand{\thesistitle}{Writing a Thesis with \LaTeX}

\title{Writing a Thesis with \LaTeX}

\author{Vorname Nachname}

\date{Februar 2020}



%************************************
% Glossaries and Acronyms
%************************************
\makeglossaries

\newacronym{snn}{SNN}{Spiking Neural Network}
\newacronym{dl}{DL}{deep learning}
\newacronym{dnn}{DNN}{deep neural network}
\newacronym{nn}{NN}{Neural Network}
\newacronym{ann}{ANN}{Artifical Neural Network}
\newacronym{lif}{LIF}{Leaky-Integrate-and-Fire}
\newacronym{hhm}{HHM}{Hodgkin-Huxley Model}
\newacronym{stdp}{STDP}{spike-timing-dependent plasticity}
\newacronym{ki}{KI}{Künstliche Intelligenz}
\newacronym{ai}{AI}{Artificial Intelligence}
\newacronym{ikt}{IKT}{Informations- und Kommunikationstechnik}
\newacronym{it}{IT}{Informationstechnik}
\newacronym{desi}{DESI}{Digital Economy and Society Index}
\newacronym{oecd}{OECD}{Organisation für wirtschaftliche Zusammenarbeit und Entwicklung}

\newglossaryentry{latex}
{
        name=latex,
        description={Is a mark up language specially suited for 
scientific documents}
}
 
\newglossaryentry{maths}
{
        name=mathematics,
        description={Mathematics is what mathematicians do}
}
 
\newglossaryentry{formula}
{
        name=formula,
        description={A mathematical expression}
}
\newglossaryentry{FaceID}
{
        name={Face-ID},
        description={Eine Authentifizierungsmethode mithilfe von Gesichtserkennung der Firma Apple}
}
\newglossaryentry{smsTAN}
{
        name={SMS-TAN},
        description={Abkürzung für Transaktionsnummer, ein elektronsiches Einmalkennwort, welches in diesem Fall über SMS versendet wird}
}
\newglossaryentry{n26}
{
        name={N26},
        description={N26 (Number 26), Bank die sich auf eBanking spezialisiert hat}
}


%************************************
% Document Starts Here
%
%
%
% Good Latex Help:
% http://www.maths.adelaide.edu.au/anthony.roberts/LaTeX/ltxxref.php
%************************************
\begin{document}



%************************************
% Title Page
%************************************
\begin{titlepage}
 \centering
% \includegraphics[width=0.15\textwidth]{example-image-1x1}\par\vspace{1cm}
 \includegraphics[height=3cm]{imgs/tu.png}
 \par\vspace{1cm}
 {\scshape\LARGE TU WIEN \par Fakultät für Elektrotechnik \par}
 \vspace{1cm}
% {\scshape\Large Final year project\par}
% \vspace{1.5cm}
 {\huge\bfseries \thesistitle \par}
% {\huge\itshape \thesissubtitle \par}
 \vspace{2cm}
 {Dimplomarbeit verfasst von\par}
 {\Large Vorname \textsc{Nachname}\par}
 {Matrikelnummer \par}
  \vspace{2cm}
 Unter der Betreuung von \par Vorname \textsc{Nachname} \par 
 \vfill
% Bottom of the page
 {\large 1. Januar 2020 \par}
\end{titlepage}

%************************************
% Front Matter
%************************************


\cleardoublepage
\section*{Abstrakt}
\lipsum[1-10]

\cleardoublepage
\include{00_erklaerung}

\cleardoublepage
\section*{Danksagung}

\lipsum[1-3]


\cleardoublepage
\tableofcontents




%************************************
% Main Content
%************************************
\cleardoublepage
\chapter{Einleitung}
\epigraph{\itshape wubba lubba dub dub!}{-- Rick Sanchez}

\section{Einleitung}

% Control the length of the epigraph with this parameters
%\setlength\epigraphwidth{.8\textwidth}
%\setlength\epigraphrule{0pt}
\subsection{Verwenden von Akronymen und Glossaren}

Diese werden vor dem Start des Dokuments definiert und können danach verwendet werden. 

Erstes Vorkommen: \gls{ai}

Zweites Vorkommen: \gls{ai}

Erzwingen von Plural: \glspl{ai} 

\subsection{Zitieren}
Es gibt mehrere Arten zu zitieren. Entweder verwendet man \autocite{adams1995hitchhiker} oder man verwendet es direkt: \cite{adams1995hitchhiker}
\lipsum[1-30]


\section{Lorem}
\lipsum[1-5]
\chapter{Hauptteil A}
\section{Section A1}
\lipsum[1-5]

\section{Section A2}
\lipsum[1-5]
\chapter{Hauptteil B}
\section{Statistiken über Österreich}
\lipsum[1-10]

\begin{table}[htbp]
     \centering
     % http://www.statistik.at/web_de/statistiken/energie_umwelt_innovation_mobilitaet/informationsgesellschaft/ikt-einsatz_in_unternehmen/022196.html
\begin{tabular}{cccccc}
\toprule
Jahr        &  Internetzugang & Fest. & Mobil & Website & Social Media \\
\midrule
2003        &   89,2	&   48,7   &    .	  &   68,4    &    .      \\
2004        &   93,9	&   55,2   &    .	  &   73,2    &    .      \\
2005        &   95,4	&   61,5   &    .	  &   72,8    &    .      \\
2006        &   97,7	&   69,8   &    .	  &   79,3    &    .      \\
2007        &   97,2	&   73,4   &    .	  &   80,5    &    .      \\
2008        &   97,1	&   76,9   &    .	  &   80,2    &    .      \\
2009        &   97,7	&   76,0   &    .	  &   80,2    &    .      \\
2010        &   97,2	&   75,5   &    46,4  &   80,6    &    .      \\
2011        &   98,2	&   82,4   &    65,1  &   82,9    &    .      \\
2012        &   98,2	&   86,4   &    57,6  &   82,0    &    .      \\
2013        &   97,6	&   85,8   &    65,7  &   85,7    &    38,6   \\
2014        &   98,4	&   91,7   &    74,4  &   86,3    &    41,4   \\
2015        &   98,8	&   90,7   &    77,1  &   87,5    &    42,0   \\
2016        &   99,0	&   92,0   &    76,2  &   88,1    &    49,5   \\
2017        &   99,7	&   91,4   &    80,5  &   85,6    &    52,9   \\
2018        &   99,6	&   90,4   &    78,1  &   87,9    &    .      \\
2019        &   99,6	&   89,6   &    79,8  &   89,5    &    59,6   \\
\bottomrule
\end{tabular}
     \caption[IKT Technologien in Unternehmen]{\gls{ikt} in Unternehmen. Zahlen in Prozent der Gesamtzahl der Unternehmen in Österreich.
     \cite{Austria2020Feb}.}
    \label{tab:ikt-unternehmen-oesterreich}
\end{table}

\lipsum[10-20]
\section{Section B2}
\lipsum{4}



%************************************
% Conclusion
%************************************
\chapter{Conclusio}
\section{Zusammenfassung}
\lipsum[1-5]



%************************************
% Appendix
%************************************
\cleardoublepage
\appendix
%\section*{Appendix}

\subsection*{Normal Verteilung}

Für eine normalverteilte stochastische Größe $x \in \mathcal{N}$ gilt die Wahrscheinlichkeitsdichtefunktion
\begin{equation}
    P(x|X) = \frac{1}{\sqrt{2\pi\sigma^2}} \int_{-\infty}^{\infty} \exp{\frac{(x-\mu)^2}{2\sigma^2}} dx
\end{equation}
mit dem Erwartungswert $\mu$ und der Standardabweichung $\sigma$.

\subsection*{Berechnung des Korrelationskoeffizienten}
The Pearson correlation coefficient is defined as:
\begin{equation}
    \rho(A, B)=\frac{1}{N-1} \sum_{i=1}^{N}\left(\frac{A_{i}-\mu_{A}}{\sigma_{A}}\right)\left(\frac{B_{i}-\mu_{B}}{\sigma_{B}}\right)
\end{equation}
where $\mu A$ and $\rho A$ are the mean and standard deviation of $A$, respectively, and $\mu B$ and $\rho B$ are the mean and standard deviation of $B$. Correlation coefficient can be written in terms of covariance of $A$ and B:

\cleardoublepage
\listoffigures
\listoftables
% Add an index.
% \printindex
% Add a glossary.
\cleardoublepage
\printglossaries
% \printglossary[type=\acronymtype]


%\bibliography{references}
%\bibliographystyle{plain}
\cleardoublepage
\printbibliography
\end{document}
